



\documentclass[journal]{IEEEtran}

% *** CITATION PACKAGES ***
%
\usepackage{cite}
\usepackage{tikz}

% *** GRAPHICS RELATED PACKAGES ***
%
\ifCLASSINFOpdf
  \usepackage[pdftex]{graphicx}
  \graphicspath{{../pdf/}{../jpeg/}}
  \DeclareGraphicsExtensions{.pdf,.jpeg,.png}
\else
  \usepackage[dvips]{graphicx}
  \graphicspath{{../eps/}}
  \DeclareGraphicsExtensions{.eps}
\fi
\usepackage{amsmath}
\interdisplaylinepenalty=2500

\usepackage{array}
\usepackage{fixltx2e}
\usepackage{stfloats}
\usepackage{url}
% Basically, \url{my_url_here}.


% correct bad hyphenation here
\hyphenation{op-tical net-works semi-conduc-tor}


\begin{document}
% Definition of blocks:
\tikzset{%
  block/.style    = {draw, thick, rectangle, minimum height = 3em,
    minimum width = 3em},
  sum/.style      = {draw, circle, node distance = 2cm}, % Adder
  input/.style    = {coordinate}, % Input
  output/.style   = {coordinate} % Output
}

\title{EEG and MEG Inversion Using Convolutional and Recurrent Neural Networks}
%
%
% author names and IEEE memberships
% note positions of commas and nonbreaking spaces ( ~ ) LaTeX will not break
% a structure at a ~ so this keeps an author's name from being broken across
% two lines.
% use \thanks{} to gain access to the first footnote area
% a separate \thanks must be used for each paragraph as LaTeX2e's \thanks
% was not built to handle multiple paragraphs
%

\author{Joaquin~J.~Casanova,~\IEEEmembership{Member,~IEEE,}
        Zachary~D.~Stoecker-Sylvia,~\IEEEmembership{Member,~IEEE,}
        Ryan~Miyamoto,~\IEEEmembership{Member,~IEEE,}
        and~Jenshan~Lin,~\IEEEmembership{Fellow,~IEEE}% <-this % stops a space
\thanks{J. Casanova and J. Lin are with the Department
of Electrical and Computer Engineering, University of Florida, Gainesville,
FL, 32611 USA e-mail: jcasa@ufl.edu}% <-this % stops a space
\thanks{R. Miyamoto and Z. Stoecker-Sylvia are with Oceanit.}% <-this % stops a space

\thanks{Manuscript received }}

% The paper headers
\markboth{Transaction on Neural Systems and Rehabilitation Engineering,~Vol.~14, No. 8}%
{Casanova \MakeLowercase{\textit{et al.}}: EEG and MEG Inversion Using Convolutional and Recurrent Neural Networks}
\maketitle

\begin{abstract}

  BCI, diagnostics - localize neural activity
  Measurement techniques dense sensors
  Average acros trials
  Typical inversion approach
  Our approach more simplified CNN/RNN/MLP
  Test data sets
  evaluate architectures for error and ability to generalize after training
  Key results
  
\end{abstract}

% Note that keywords are not normally used for peerreview papers.
\begin{IEEEkeywords}
EEG, MEG, Localization, Neural networks.
\end{IEEEkeywords}






% For peer review papers, you can put extra information on the cover
% page as needed:
% \ifCLASSOPTIONpeerreview
% \begin{center} \bfseries EDICS Category: 3-BBND \end{center}
% \fi
%
% For peerreview papers, this IEEEtran command inserts a page break and
% creates the second title. It will be ignored for other modes.
\IEEEpeerreviewmaketitle



\section{Introduction}

\IEEEPARstart{T}{here} is a great need for interpretation of brain signals for both use in control of devices, for prosthetics, for example, or for disease diagnostics \cite{}. Sensor measurements include ...
Problem of neuron localization or distribution of currents
typical approaches
our approach: max dipole

% You must have at least 2 lines in the paragraph with the drop letter
% (should never be an issue
\cite{gramfort2013meg}

\section{Methods}
\subsubsection{Datasets}
Subsubsection text here.
Audio
Faces
\subsection{Precprocessing}
\subsection{Description of Neural Networks}

Subsection text here.

% needed in second column of first page if using \IEEEpubid
%\IEEEpubidadjcol
\subsection{Hyperparameters}

\subsection{Training and testing}
\section{Results}

\section{Conclusion}
The conclusion goes here.


% use section* for acknowledgment
\section*{Acknowledgment}


The authors would like to thank...



% references section

\bibliographystyle{IEEEtran}
\bibliography{./casanova_ieee}
%\begin{IEEEbiography}[{\includegraphics[width=1in,height=1.25in,clip,keepaspectratio]{mshell}}]{Michael Shell}
% or if you just want to reserve a space for a photo:

\begin{IEEEbiography}{}
\end{IEEEbiography}

% if you will not have a photo at all:
\begin{IEEEbiographynophoto}{}
\end{IEEEbiographynophoto}

% insert where needed to balance the two columns on the last page with
% biographies
%\newpage

\begin{IEEEbiographynophoto}{Jane Doe}
Biography text here.
\end{IEEEbiographynophoto}

% You can push biographies down or up by placing
% a \vfill before or after them. The appropriate
% use of \vfill depends on what kind of text is
% on the last page and whether or not the columns
% are being equalized.

%\vfill

% Can be used to pull up biographies so that the bottom of the last one
% is flush with the other column.
%\enlargethispage{-5in}



% that's all folks
\end{document}


\grid
