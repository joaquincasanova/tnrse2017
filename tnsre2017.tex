



\documentclass[journal]{IEEEtran}

% *** CITATION PACKAGES ***
%
\usepackage{cite}

% *** GRAPHICS RELATED PACKAGES ***
%
\ifCLASSINFOpdf
  \usepackage[pdftex]{graphicx}
  \graphicspath{{../pdf/}{../jpeg/}}
  \DeclareGraphicsExtensions{.pdf,.jpeg,.png}
\else
  \usepackage[dvips]{graphicx}
  \graphicspath{{../eps/}}
  \DeclareGraphicsExtensions{.eps}
\fi
\usepackage{amsmath}
\interdisplaylinepenalty=2500

\usepackage{array}
\usepackage{fixltx2e}
\usepackage{stfloats}
\usepackage{url}
% Basically, \url{my_url_here}.


% correct bad hyphenation here
\hyphenation{op-tical net-works semi-conduc-tor}


\begin{document}
% Definition of blocks:

\title{EEG and MEG Inversion Using Convolutional and Recurrent Neural Networks}
%
%
% author names and IEEE memberships
% note positions of commas and nonbreaking spaces ( ~ ) LaTeX will not break
% a structure at a ~ so this keeps an author's name from being broken across
% two lines.
% use \thanks{} to gain access to the first footnote area
% a separate \thanks must be used for each paragraph as LaTeX2e's \thanks
% was not built to handle multiple paragraphs
%

\author{Joaquin~J.~Casanova,~\IEEEmembership{Member,~IEEE,}
        Zachary~D.~Stoecker-Sylvia,~\IEEEmembership{Member,~IEEE,}
        Ryan~Miyamoto,~\IEEEmembership{Member,~IEEE,}
        and~Jenshan~Lin,~\IEEEmembership{Fellow,~IEEE}% <-this % stops a space
\thanks{J. Casanova and J. Lin are with the Department
of Electrical and Computer Engineering, University of Florida, Gainesville,
FL, 32611 USA e-mail: jcasa@ufl.edu}% <-this % stops a space
\thanks{R. Miyamoto and Z. Stoecker-Sylvia are with Oceanit.}% <-this % stops a space

\thanks{Manuscript received }}

% The paper headers
\markboth{Transaction on Neural Systems and Rehabilitation Engineering,~Vol.~XX, No. YY}%
{Casanova \MakeLowercase{\textit{et al.}}: EEG and MEG Dipole Localization Using Convolutional and Recurrent Neural Networks}
\maketitle

\begin{abstract}

  Real-time localzation of neuronal activity has a number of uses, including brain-computer interfaces and medical diagnostics. Generally, this is done by taking measurements of the brain's magnetic and electric fields (magnetoencephalography [MEG] and electroencephalography [EEG]), and inverting the measurements. Most approaches are physically-based, and attempt to find the best estimate of the lead-field matrix, which relates the dipole activation to the field strength by minimizing the error of an estimate. To date, most approaches using the lead-field matrix are complicated and too slow in real time, aimed primarily at elucidating brain structural functional relationships from experimental data. We propose a new technique in which the location of peak neuronal current is estimated by treating EEG and MEG as a two-channel image, or time-series of images, which is processed by a neural network which returns the location of the dipole of peak magnitude. Four archtectures are tested: 2-layer perceptron, convolutional neural network (CNN), recurrent neural network (RNN), and CNN feeding RNN. In the absence of true measures of neuronal activity, we used two publicly available MEG/EEG datasets, and treated the estimates of the traditional minimum-norm estimate (MNE) as true estimates. We test the four variations of the network architecture, and in the best case (CNN only), we achieve test dataset errors (RMSE of max dipole location) of between XX and YY mm.
  
\end{abstract}

% Note that keywords are not normally used for peerreview papers.
\begin{IEEEkeywords}
EEG, MEG, Localization, Neural networks.
\end{IEEEkeywords}






% For peer review papers, you can put extra information on the cover
% page as needed:
% \ifCLASSOPTIONpeerreview
% \begin{center} \bfseries EDICS Category: 3-BBND \end{center}
% \fi
%
% For peerreview papers, this IEEEtran command inserts a page break and
% creates the second title. It will be ignored for other modes.
\IEEEpeerreviewmaketitle



\section{Introduction}

\IEEEPARstart{T}{here} is a great need for interpretation of brain signals for both use in control of devices, for prosthetics, for example, or for disease diagnostics . Sensor measurements include ...
Problem of neuron localization or distribution of currents
typical approaches
our approach: max dipole

\cite{mellinger2007meg}

An MEG-based brain--computer interface (BCI)

\cite{da2008impact}

The impact of EEG/MEG signal processing and modeling in the diagnostic and management of epilepsy

Inversion methods

\cite{grech2008review}

Review on solving the inverse problem in EEG source analysis

\cite{mosher1992multiple}

Multiple dipole modeling and localization from spatio-temporal MEG data

\cite{buchner1997inverse}

Inverse localization of electric dipole current sources in finite element models of the human head

\cite{galka2004solution}

A solution to the dynamical inverse problem of EEG generation using spatiotemporal Kalman filtering

\cite{somersalo2003non}

Non-stationary magnetoencephalography by Bayesian filtering of dipole models

\cite{taulu2005applications}

Applications of the signal space separation method

\cite{sekihara2001reconstructing}

Reconstructing spatio-temporal activities of neural sources using an MEG vector beamformer technique


\cite{van2000eeg}

EEG dipole source localization using artificial neural networks

Neural nets

\cite{lecun1998gradient}

Convolutional


\cite{hochreiter1997long}

Long short-term memory
Recurrent

\cite{venugopalan2014translating}
Translating videos to natural language using deep recurrent neural networks

% You must have at least 2 lines in the paragraph with the drop letter
  % (should never be an issue

Data/processing
\cite{gramfort2013meg}

\section{Methods}
\subsubsection{Datasets}
\cite{gramfort2013meg}
Subsubsection text here.
Audio
Faces
\subsection{Preprocessing}
\subsection{Description of Neural Networks}
\url{tensorflow.com}
\cite{kingma2014adam}

Adam: A method for stochastic optimization

\begin{figure*}[!t]
\centering
\includegraphics[width=7in]{cnnrnn}
% where an .eps filename suffix will be assumed under latex, 
% and a .pdf suffix will be assumed for pdflatex; or what has been declared
% via \DeclareGraphicsExtensions.
\caption{Block diagram of CNN+RNN neural network.}
\label{fig_cnnrnn}
\end{figure*}
\begin{figure*}[!t]
\centering
\includegraphics[width=7in]{cnn}
% where an .eps filename suffix will be assumed under latex, 
% and a .pdf suffix will be assumed for pdflatex; or what has been declared
% via \DeclareGraphicsExtensions.
\caption{Block diagram of CNN neural network.}
\label{fig_cnn}
\end{figure*}
\begin{figure*}[!t]
\centering
\includegraphics[width=7in]{rnn}
% where an .eps filename suffix will be assumed under latex, 
% and a .pdf suffix will be assumed for pdflatex; or what has been declared
% via \DeclareGraphicsExtensions.
\caption{Block diagram of RNN neural network.}
\label{fig_rnn}
\end{figure*}
\begin{figure*}[!t]
\centering
\includegraphics[width=7in]{mlp}
% where an .eps filename suffix will be assumed under latex, 
% and a .pdf suffix will be assumed for pdflatex; or what has been declared
% via \DeclareGraphicsExtensions.
\caption{Block diagram of MLP neural network.}
\label{fig_mlp}
\end{figure*}

Subsection text here.

% needed in second column of first page if using \IEEEpubid
%\IEEEpubidadjcol
\subsection{Hyperparameters}

\subsection{Training and testing}
\section{Results}

\begin{figure*}[!t]
\centering
\includegraphics[width=7in]{aud}
% where an .eps filename suffix will be assumed under latex, 
% and a .pdf suffix will be assumed for pdflatex; or what has been declared
% via \DeclareGraphicsExtensions.
\caption{Training/validation results for auditory stimulus dataset.}
\label{fig_aud}
\end{figure*}

\begin{figure*}[!t]
\centering
\includegraphics[width=7in]{faces}
% where an .eps filename suffix will be assumed under latex, 
% and a .pdf suffix will be assumed for pdflatex; or what has been declared
% via \DeclareGraphicsExtensions.
\caption{Training/validation results for faces stimulus dataset.}
\label{fig_faces}
\end{figure*}

\section{Conclusion}
The conclusion goes here.


% use section* for acknowledgment
\section*{Acknowledgment}


The authors would like to thank...



% references section

\bibliographystyle{IEEEtran}
\bibliography{./casanova_ieee}
%\begin{IEEEbiography}[{\includegraphics[width=1in,height=1.25in,clip,keepaspectratio]{mshell}}]{Michael Shell}
% or if you just want to reserve a space for a photo:

\begin{IEEEbiography}{}
\end{IEEEbiography}

% if you will not have a photo at all:
\begin{IEEEbiographynophoto}{}
\end{IEEEbiographynophoto}

% insert where needed to balance the two columns on the last page with
% biographies
%\newpage

\begin{IEEEbiographynophoto}{Jane Doe}
Biography text here.
\end{IEEEbiographynophoto}

% You can push biographies down or up by placing
% a \vfill before or after them. The appropriate
% use of \vfill depends on what kind of text is
% on the last page and whether or not the columns
% are being equalized.

%\vfill

% Can be used to pull up biographies so that the bottom of the last one
% is flush with the other column.
%\enlargethispage{-5in}



% that's all folks
\end{document}


\grid
